\subsubsection{Wetlands or Internally Draining Areas}\label{sec:wetlands}

	SWAT considers wetlands in a manner very similar to how it considers ponds, the difference only being in the outflow calculation. That said there were several parameters that needed to be calculated for the basin's wetlands and these were the same as for ponds: the fraction of each subbasin composed of wetlands, the normal and maximum surface areas, and the normal and maximum volumes. 
	
	These parameters were calculated using a topography-based approach and were calculated by individual subbasin. A digital elevation model (DEM) was filled using a the Fill function in ArcGIS, filling all of the sink areas and causing all simulated water to run off of the landscape. The original DEM was subtracted from the filled DEM to derive a surface of the depth of internally drained areas or sinks; this sinks layer shows the internally drained areas for the basin. The sink layer provided the starting point for the wetlands layer. 
	
	The areas identified by the USDA-NASS's Cropland Data Layer (CDL) as herbaceous and woody wetlands and cranberry were considered to be areas where wetland vegetation is likely to be found. If wetland vegetation exists it can be assumed that that landscape has a consistent wetland hydrology, consistent enough that it is expressed in the vegetation. The intersection or overlap of the sinks layer and the wetland vegetation, as identified by the CDL, was considered to be the normal wetland area. From this the normal surface area was calculated. The depth of the sinks were used as the depth of water in the wetland areas and this was multiplied by the normal surface area to calculate the normal volume. For maximum surface area, all sinks were considered to be wetland areas, regardless of their relationship with the wetland vegetation map. The area of the sinks was considered the maximum surface area. The maximum surface area was multiplied by the sink depth to derive the maximum volume. The maximum wetland surface area was divided by the subbasin area to derive the fraction of the subbasin that is wetland.
	
	There are precedents to using a topography-based approach to defining wetland areas in SWAT; several studies conducted in the midwest are discussed here. \citet{almendinger_constructing_2007} considered internally drained areas as wetlands (as identified by remote sensing\footnote{Specifically, the remotely sensed imagery was from the WISCLAND data set; a dataset of landcover determined from LANDSAT imagery.}) if they were not connected to the main channel and lakes were considered ponds in their SWAT model. Wetlands, identified through remote sensing, were considered SWAT wetlands only if they occur on the main channel. Similarly, \citet{kirsch_predicting_2002} considered internally drained areas as wetlands in SWAT if they overlapped with remotely-sensed-defined wetlands; if they did not, they were considered ponds.  \citet{almendinger_constructing_2010} modeled closed internal depressions as wetlands and open (those draining to the main channel) as ponds.



